\documentclass[12pt]{article}
\usepackage{circuitikz}
\usepackage{tikz}

\begin{document}
\tikzstyle{block} = [draw, rectangle, minimum height=3em, minimum width=6em]

\section*{Input mixer}

The input stage contains of a small mixer with two potentiometers;
one mix between input 1 and input 2 and the other is the volume for
input 3 that also goes through an overdrive.

\bigskip

\begin{tikzpicture}
    \draw (4,3) node[adder](sum) {};
    \node [block] at (4,1) (overdrive) {Overdrive};

    \draw (0,6) node[above]{$input1$} -- (1,6) to[pR, name=mix] (1,4);
    \draw (0,4) node[above]{$input2$} -- (1,4);
    \draw (mix.wiper) -- (4,5) [->] to(sum.north);

    \draw (0,2) node[above]{$input3$} -- (1,2) to[pR, name=vol] (1,0) node[sground]{};
    \draw (vol.wiper) [->] to(overdrive.west);
    \draw (overdrive.north) [->] to(sum.south);

    \draw (sum.east) -- (6,3) [->] node[above]{$to filters$};
\end{tikzpicture}

\pagebreak

\section*{Filter(s)}

Right now the idea is to have a filter chain with a pre-processing filter,
a dual filter that can be configured in different ways and a post-processing
filter. This is, of course, not definitely settled. Yet.

\bigskip

\begin{tikzpicture}
    \node [block] at (2,0) (vcf1) {VCF1};
    \node [block] at (6,0) (vcf2) {VCF2/VCF3};
    \node [block] at (10,0) (vcf4) {VCF4};

    \draw (0,0) node[above]{$input$} [->] to(vcf1.west);
    \draw (vcf1.east) -- (vcf2.west) [->];
    \draw (vcf2.east) -- (vcf4.west) [->];
    \draw (vcf4.east) -- (12,0) [->] node[above]{$output$};
\end{tikzpicture}

\bigskip

The filter is based on AS/CEM3350, which is a dual 2 pole filter, together with
s-load of MUXes that allows for reconfiguration of the setup. All the filters
has MUXes that allows them to be switched between LP/BP/HP, the default configuration
is BP, if the left MUX is activated it's turned into a LP-filter and the right turns
it into a HP.

\bigskip

\begin{tikzpicture}
    \node [cute spdt down] at (1,0) (lp) {};
    \node [cute spdt down, xscale=-1] at (11,0) (hp) {};
    \node [block] at (6,0) (vcf) {VCF}
    ($(vcf.west)!0.68!(vcf.north west)$) coordinate (vcfina)
    ($(vcf.west)!0.68!(vcf.south west)$) coordinate (vcfinb)
    ($(vcf.east)!0.68!(vcf.north east)$) coordinate (vcfouta)
    ($(vcf.east)!0.68!(vcf.south east)$) coordinate (vcfoutb);

    \draw (0,0) node[above]{$input$} to(lp.in);
    \draw (lp.out 1) -- (vcfina) node[midway,above]{$LP$};
    \draw (lp.out 2) -- (vcfinb) node[midway,above]{$BP/HP$};
    \draw (hp.out 1) -- (vcfouta) node[midway,above]{$HP$};
    \draw (hp.out 2) -- (vcfoutb) node[midway,above]{$LP/BP$};
    \draw (hp.in) -- (12,0) [->] node[above]{$output$};
\end{tikzpicture}

\bigskip

The VCF2 and VCF3 creates a filter block where the filters can run in parallel,
serial or feedback mode.

\pagebreak

\subsection*{Parallel mode}

In this mode both VFC2 and VCF3 are fed the signal from the input and the
mixer potentiometer mix between the VCF outputs.

\bigskip

\begin{tikzpicture}
    % Signal path pot
    \node [cute spdt down] at (6,2) (potsw) {};
    \draw (potsw.in) -- (5,2) to[pR, name=mix] (5,0);

    % Signal path output
    \node [cute spdt up] at (6,-3) (outsw) {};
    \draw (6.35,1) -- (outsw.out 1);
    \draw (outsw.in) -- (5,-3) -- (5, -4) [->] node[below]{$output$};

    % Signal path Overdrive
    \node [block] at (3,0) (od) {Overdrive};
    \node [cute spdt up] at (1,0) (odsw) {};
    \draw (3,1) -- (1.35,1) -- (odsw.out 1);
    \draw (od.south) -- (3,-1) -- (1.35,-1) -- (odsw.out 2);
    \draw (odsw.in) -- (0.65,-1.5) -- (5,-1.5) -- (mix.east);

    % Signal path Inverter
    \node [block] at (3,3) (inv) {Inverter};
    \node [cute spdt up] at (1,3) (invsw) {};
    \draw (3,4) -- (1.35,4) -- (invsw.out 1);
    \draw (inv.south) -- (3,2) -- (1.35,2) -- (invsw.out 2);
    \draw (invsw.in) -- (0.65,1.5) -- (3,1.5) -- (od.north) [->];

    % Signal path VCF2
    \node [block] at (3,6) (vcf2) {VCF2};
    \node [cute spdt up] at (3.35,9) (in2sw) {};
    \draw (in2sw.in) -- (vcf2.north) [->];
    \draw (vcf2.south) -- (inv.north) [->];
    \draw (8,5) -- (6.45,5);
    \draw (6.25,5) -- (5,5) -- (5,8.55) -- (in2sw.out 2);

    % Signal path VCF3
    \node [block] at (8,6) (vcf3) {VCF3};
    \node [cute spdt up] at (8.35,9) (in3sw) {};
    \draw (in3sw.in) -- (vcf3.north) [->];
    \draw (vcf3.south) -- (8, -3.42) -- (outsw.out 2);
    \draw (8,1.57) -- (potsw.out 2);
    \draw (mix.wiper) -- (7.9,1);
    \draw (8.1,1) -- (10,1) -- (10,8.55) -- (in3sw.out 2);

    % Signal path input
    \draw (6.35,11) node[above]{$input$} -- (potsw.out 1);
    \draw (in2sw.out 1) -- (3.71,10) -- (8.71,10) -- (in3sw.out 1);
\end{tikzpicture}

\pagebreak

\subsection*{Serial mode}

In this mode the input goes to VCF2 and the potentiometer works as a wet/dry
mix that feeds the input of VCF3. Output is taken directly from VCF3.

\bigskip

\begin{tikzpicture}
    % Signal path pot
    \node [cute spdt up] at (6,2) (potsw) {};
    \draw (potsw.in) -- (5,2) to[pR, name=mix] (5,0);

    % Signal path output
    \node [cute spdt down] at (6,-3) (outsw) {};
    \draw (6.35,1) -- (outsw.out 1);
    \draw (outsw.in) -- (5,-3) -- (5, -4) [->] node[below]{$output$};

    % Signal path Overdrive
    \node [block] at (3,0) (od) {Overdrive};
    \node [cute spdt up] at (1,0) (odsw) {};
    \draw (3,1) -- (1.35,1) -- (odsw.out 1);
    \draw (od.south) -- (3,-1) -- (1.35,-1) -- (odsw.out 2);
    \draw (odsw.in) -- (0.65,-1.5) -- (5,-1.5) -- (mix.east);

    % Signal path Inverter
    \node [block] at (3,3) (inv) {Inverter};
    \node [cute spdt up] at (1,3) (invsw) {};
    \draw (3,4) -- (1.35,4) -- (invsw.out 1);
    \draw (inv.south) -- (3,2) -- (1.35,2) -- (invsw.out 2);
    \draw (invsw.in) -- (0.65,1.5) -- (3,1.5) -- (od.north) [->];

    % Signal path VCF2
    \node [block] at (3,6) (vcf2) {VCF2};
    \node [cute spdt up] at (3.35,9) (in2sw) {};
    \draw (in2sw.in) -- (vcf2.north) [->];
    \draw (vcf2.south) -- (inv.north) [->];
    \draw (8,5) -- (6.45,5);
    \draw (6.25,5) -- (5,5) -- (5,8.55) -- (in2sw.out 2);

    % Signal path VCF3
    \node [block] at (8,6) (vcf3) {VCF3};
    \node [cute spdt down] at (8.35,9) (in3sw) {};
    \draw (in3sw.in) -- (vcf3.north) [->];
    \draw (vcf3.south) -- (8, -3.42) -- (outsw.out 2);
    \draw (8,1.57) -- (potsw.out 2);
    \draw (mix.wiper) -- (7.9,1);
    \draw (8.1,1) -- (10,1) -- (10,8.55) -- (in3sw.out 2);

    % Signal path input
    \draw (6.35,11) node[above]{$input$} -- (potsw.out 1);
    \draw (in2sw.out 1) -- (3.71,10) -- (8.71,10) -- (in3sw.out 1);
\end{tikzpicture}

\pagebreak

\subsection*{Feedback mode}

In this mode the input goes to the potentiometer that mix between input and
the output from VCF2, this is then fed into VCF3 which output then goes out
from the filterblock and is also fed into VCF2. VCF2 therefore works like a
filtered feedback for VCF3.

\begin{tikzpicture}
    % Signal path pot
    \node [cute spdt up] at (6,2) (potsw) {};
    \draw (potsw.in) -- (5,2) to[pR, name=mix] (5,0);

    % Signal path output
    \node [cute spdt down] at (6,-3) (outsw) {};
    \draw (6.35,1) -- (outsw.out 1);
    \draw (outsw.in) -- (5,-3) -- (5, -4) [->] node[below]{$output$};

    % Signal path Overdrive
    \node [block] at (3,0) (od) {Overdrive};
    \node [cute spdt up] at (1,0) (odsw) {};
    \draw (3,1) -- (1.35,1) -- (odsw.out 1);
    \draw (od.south) -- (3,-1) -- (1.35,-1) -- (odsw.out 2);
    \draw (odsw.in) -- (0.65,-1.5) -- (5,-1.5) -- (mix.east);

    % Signal path Inverter
    \node [block] at (3,3) (inv) {Inverter};
    \node [cute spdt up] at (1,3) (invsw) {};
    \draw (3,4) -- (1.35,4) -- (invsw.out 1);
    \draw (inv.south) -- (3,2) -- (1.35,2) -- (invsw.out 2);
    \draw (invsw.in) -- (0.65,1.5) -- (3,1.5) -- (od.north) [->];

    % Signal path VCF2
    \node [block] at (3,6) (vcf2) {VCF2};
    \node [cute spdt down] at (3.35,9) (in2sw) {};
    \draw (in2sw.in) -- (vcf2.north) [->];
    \draw (vcf2.south) -- (inv.north) [->];
    \draw (8,5) -- (6.45,5);
    \draw (6.25,5) -- (5,5) -- (5,8.55) -- (in2sw.out 2);

    % Signal path VCF3
    \node [block] at (8,6) (vcf3) {VCF3};
    \node [cute spdt down] at (8.35,9) (in3sw) {};
    \draw (in3sw.in) -- (vcf3.north) [->];
    \draw (vcf3.south) -- (8, -3.42) -- (outsw.out 2);
    \draw (8,1.57) -- (potsw.out 2);
    \draw (mix.wiper) -- (7.9,1);
    \draw (8.1,1) -- (10,1) -- (10,8.55) -- (in3sw.out 2);

    % Signal path input
    \draw (6.35,11) node[above]{$input$} -- (potsw.out 1);
    \draw (in2sw.out 1) -- (3.71,10) -- (8.71,10) -- (in3sw.out 1);
\end{tikzpicture}

\end{document}
